\documentclass[10pt]{beamer}

\usepackage[frenchb]{babel}
% \usepackage[latin1]{inputenc}
\usepackage[utf8]{inputenc}
\usepackage{amsmath}
\usepackage{amssymb}
\usepackage{latexsym}
\usepackage{graphicx}
\usepackage{ucs}
\usepackage{mathrsfs}


% Déclartion des commandes
\newcommand{\reels}{\mathbb{R}}
\newcommand{\integers}{\mathbb{Z}}
\newcommand{\Lagr}{\mathcal{L}}
\newcommand\norm[1]{\left\lVert#1\right\rVert}

\usetheme{Warsaw}

\title{Description d'article}
% \subtitle{A Progressive Hedging Based Branch-and-Bound Algorithm for Stochastic Mixed-Integer Programs}
\author{Armand Fouquiau, Romany Stéphane}
\institute{Université Paris-Sud}
\date{Octobre 2017}

% Debut - Rédaction du diaporama
\begin{document}

    \begin{frame}
    \titlepage
        % Une page de présentation
    \end{frame}

    \begin{section}{Problématique}
        \begin{frame}
        \frametitle{But de l'article}
        % Présenter le but de l'article
        Cette article a pour but de de combler le vide littéraire concernant une famille de programmes stochastique: Stochastic Mixed-Integer Convexe Programs.
    \end{frame}
        
    \begin{frame}
        \frametitle{But de l'article}
        Toutes les décisions prises pour chaques scénarios à l'instant $t-1$ sont prises en compte à l'instant $t$. La fonction d'association $X(\xi_s) = x_s$ implique que $x_s$ dépend des variables aléatoires de la distribution $\xi$.
    \end{frame}
    
    
    \begin{frame}
        \frametitle{But de l'article}
        La prise de décision se fait suivant des scénarios des sous problèmes de la forme : 
         $f(x_s, \;\xi_{s}) = min \;\{f_s(x_s) \; | \; x_s \in C_s, \,x_s \in \reels^{n_r} \times \integers^{n_z}\}$\\
         Avec $n = n_r + n_z$, la taille du vecteur $x_s$
    \end{frame}    
    
    \begin{frame}
        \frametitle{But de l'article}               
    \end{frame}
    \end{section}
    
    \begin{section}{Etat de l'art}
    \begin{frame}
        \frametitle{Etat de l'art}
        %Présenter la methode PHA
        L'algorithme Progressive Hedging (Haies progressive) a été inventer en 1991 par R. Rockafellar et West, 
    \end{frame}
    
    \begin{frame}
        \frametitle{Etat de l'art}
        %Présenter la methode PHA
        L'algorithme Progressive Hedging (Haies progressive) a été inventer en 1991 par R. Rockafellar et West, 
    \end{frame}
    \end{section}
        
    \begin{section}{Approches/Méthodes Etudiées}
    
    \begin{frame}
        \frametitle{Généralités}
        Si on relaxe l'ensemble des contraintes de nonaticipativité noté ici $\mathcal{N}$, On peut convertir le problème P en problème convexe.
    \end{frame}
    
    \begin{frame}
        \frametitle{Généralités}
        Accordingly, letting $\mathcal{Y}$\\ % Y cursive
        represent the set of feasible dual multipliers, the ordinary Lagrangian, achieved through the dualization
of the constraint $X - \hat{X} = 0$.
    \end{frame}
    
    
    \begin{frame}
        \frametitle{Généralités}
        On introduit des multiplicateurs Lagrangiens $\lambda_s \, \forall s = 1 \ldots S$\\
        En multipliant ces multiplicateurs par les probabilités $p_s$ pour chaque scénario s, on obtient $p_s\lambda_s$. Ces valeurs doivent être interpretées comme les multiplicateurs duals des contraintes de nonanticipativité accocié au scénario s.\\
        La somme de tous les coefficients $p_s\cdot\lambda_s = 0$
    \end{frame}
    
    \begin{frame}
        \frametitle{Généralités}
        On interprète l'équation (1) comme une extention du Lagrangien $\Lagr(X, \hat{X}, \Lambda)$.\\
        %\begin{equation}
            
        %\end{equation}
        Avec les termes:\\
        $\norm{x_s - \hat{x}_s}^{2}_{2}$ qui permet une meilleur prise de décision,\\
        $\norm{\lambda_s - \lambda^{k-1}_{s}}^{2}_{2}$ avec k, un itération de l'algorithme PH,\\
        $\rho$, une constante positive qui accumule l'impacte des deux précédentes quantités, au fil des itérations de l'algorithme PH.
    \end{frame}
    
    \begin{frame}
        \frametitle{Généralité}
        L'algorithme PH règle certains problèmes de séparabilités due à la présence des termes $\hat{x}_s$
    \end{frame}
    
    \begin{frame}
        %\frametitle{Approches/Méthodes Etudiées}
        Les optimisations peuvent être faites pour chaque scénario indépendant, au regard de chaque variable $(X, \hat{X}, \Lambda)$
    \end{frame}
    
    
    
    \begin{frame}
        \frametitle{Progressive hedging}
        On instaure $\hat{x}^0$ en tant que minimum de la fonction objective $f(x_s, \; \Xi_s)$. Cela leur permet d'obtenir une borne inférieure.
        
    \end{frame}
    
    \begin{frame}
        \frametitle{Progressive hedging}
         Ici mettre le code de l'algo.
        
            %\begin{algo1}
            %\end{algo1}
        
    \end{frame}
    
    \begin{frame}
       L'heuristque PH peut trouver des valerus qui ne respecte pas les contraintes de nonanticipativité, alors que l'algorithme PH-BAB écarte les solutions ne respectant pas ses dernières.\\
       Pour chaque nouveaux noeuds crées, $\sigma^+,\;\sigma^-$ on garde la même contrainte de réalisabilité : $\mathrm{Q}^\sigma = \mathrm{Q}^{\sigma^+} = \mathrm{Q}^{\sigma^-}$
    \end{frame}
    
    \begin{frame}
      Les meilleurs solvers du marché peuvent détecter la consistance de $C_s^\sigma$ c'est à dire calculer en temps fini 
      $\mathcal{N}\cap(\times_{s = 1}^s C_s^\sigma)$ pour déterminer la réalisabilité de chaque relaxations de la fonction objective $z(\sigma)$
      
      Avec $C_s^\sigma \Leftrightarrow C_s \cap Q_s^\sigma$ : l'ensemble des $x_s$ réalisables pour chaque scénario individuel.
    \end{frame}
    \end{section}
    
    
    \begin{frame}
        \frametitle{}            
    \end{frame}
        
    \begin{section}{Résultats}
    \begin{frame}
    
    \end{frame}
    \end{section}
    
    
    \begin{frame}
        \frametitle{Conclusion/Perspective}
    \end{frame}
    
    \iffalse
    \fi
    % Fin de cette page de présentation

\end{document}

% Fin - Rédaction du diaporama
