\documentclass[10pt]{beamer}

\usepackage[frenchb]{babel}
% \usepackage[latin1]{inputenc}
\usepackage[utf8]{inputenc}
\usepackage{amsmath}
\usepackage{amssymb}
\usepackage{latexsym}
\usepackage{graphicx}
\usepackage{ucs}


% Déclartion des commandes
\newcommand{\reels}{\mathbb{R}}
\newcommand{\integers}{\mathbb{Z}}

\usetheme{Warsaw}

\title{Description d'article}
% \subtitle{A Progressive Hedging Based Branch-and-Bound Algorithm for Stochastic Mixed-Integer Programs}
\author{Armand Fouquiau, Romany Stéphane}
\institute{Université Paris-Sud}
\date{Octobre 2017}

% Debut - Rédaction du diaporama
\begin{document}

    \begin{frame}
    \titlepage
        % Une page de présentation
    \end{frame}

    
        \begin{frame}
        \frametitle{Problématique}
        % Présenter le but de l'article
        Cette article a pour but de de combler le vide littéraire concernant une famille de programmes stochastique: Stochastic Mixed-Integer Convexe Programs.
    \end{frame}
        
    \begin{frame}
        \frametitle{Problématique}
        Toutes les décisions prises pour chaques scénarios à l'instant $t-1$ sont prises en compte à l'instant $t$. La fonction d'association $X(\xi_s) = x_s$ implique que $x_s$ dépend des variables aléatoires de la distribution $\xi$.
    \end{frame}
    
    
    \begin{frame}
        \title{Problématique}
        La prise de décision se fait suivant des scénarios des sous problèmes de la forme : 
         $f(x_s, \xi_{s}) = min \{f_s(x_s)\} | x_s \in C_s, x_s \in \reels^{n_r} \times \integers^{n_z}$\\
         Avec $n = n_r + n_z$, la taille du vecteur $x_s$
    \end{frame}
    
    
    \begin{frame}
        \frametitle{Problématique}
        %Présenter la methode PHA
        
    \end{frame}
        
    \begin{frame}
        \frametitle{Etat de l'art}
        
    \end{frame}
        
    \begin{frame}
        \frametitle{Approches/Méthodes Etudiées}
    \end{frame}
                
    \begin{frame}
        \frametitle{Master-Worker Parallel with Barrier}            
    \end{frame}
        
    \begin{frame}
        \frametitle{Résultat}
    \end{frame}
        
    \begin{frame}
        \frametitle{Conclusion/Perspective}
    \end{frame}
    
    \iffalse
    \fi
    % Fin de cette page de présentation

\end{document}

% Fin - Rédaction du diaporama
